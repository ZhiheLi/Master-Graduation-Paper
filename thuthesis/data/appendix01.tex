\chapter{依期望采样的证明}
\label{cha:proof}

这里,我们将证明在允许一个广告被投放于同一PV两次的情况下,采样的期望等于计算所得的期望。

因为我们已经约束了$0 \le \bm{X}_{ij} \le 1$,因此划分操作可以很容易保证$\bm{p}$中的元素最多只被切分两次。于是现在只剩两种可能:
\begin{enumerate}[1)]
	\item $\bm{p}_j$没被切分。
	
	这种情况是平凡的,因为显然广告$j$被采样的期望就是$\bm{p}_j$。
	
	\item $\bm{p}_j$被切分。
	
	假设$\bm{p}_j$被分成两份:$p_1$和$p_2$。表\ref{tab:prob}展现了广告$j$被采样次数的概率,最后一行的期望是$p_1+p_2$,而这正好就是$\bm{p}_j$。
	\begin{table}[htb]
		\centering
		\caption{广告$j$被采样次数的概率(无限制)}
		\label{tab:prob}
		\begin{tabular}{cccc}
			\toprule
			被采样次数&0&1&2\\
			\midrule
			概率 & $ (1-p_1) (1-p_2)$ & $p_1 (1-p_2) +  (1-p_1)p_2$ & $p_1p_2$ \\
			\midrule
			期望 & \multicolumn{3}{c}{$p_1 (1-p_2) +  (1-p_1)p_2 + 2p_1p_2 = p_1 + p_2$} \\
			\bottomrule
		\end{tabular}
	\end{table}

\end{enumerate}

总之,如果一个广告可以被投放于同一个PV多于一次,则广告被采样,或者说被投放,的期望就是$\bm{p}$。

但是,如果广告$j$已经根据$p_1$被采样了,那么$p_2$就会被忽略的话,概率和期望就会变成表\ref{tab:prob_less}。此时被采样的期望不再等于$\bm{p}_j$,而是比$\bm{p}_j$少了$p_1p_2$。不过由于只有最多$m-1$个$\bm{p}_j$会被切分,这启发了我们最好切分那些$\bm{p}_j$较小的广告,从而减少采样的期望和计算所得期望的差距。

\begin{table}[htb]
	\centering
	\caption{广告$j$被采样次数的概率(最多被采样一次)}
	\label{tab:prob_less}
	\begin{tabular}{ccc}
		\toprule
		被采样次数&0&1\\
		\midrule
		概率 & $ (1-p_1) (1-p_2)$ & $p_1 +  (1-p_1)p_2$ \\
		\midrule
		期望 & \multicolumn{2}{c}{$p_1 + p_2 - p_1p_2$} \\
		\bottomrule
	\end{tabular}
\end{table}


