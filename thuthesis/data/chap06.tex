\chapter{结论与展望}
\label{cha:conclusion}

\section{主要工作及贡献}

随着智能手机的普及和移动互联网的快速发展,定位问题及其后续应用吸引了越来越多的关注。虽然在相对空旷的室外场景下,GPS等基于卫星定位的定位方式已经能取得较高的定位精度,但是如何在遮挡严重的城市内以及室内定位目前还处在探索和发展阶段,而且如何解决GPS解决不了的发射源定位也是一个研究点。本文针对这两个发射源定位和接收端定位的问题,分别提出了几何模型发射源定位算法和基于曼哈顿切线距离的指纹定位算法,不仅只使用不需要额外硬件支持的接收信号强度作为特征,而且相比于经典算法提高了定位精度。同时,在获得用户位置后据此向用户推送广告是一种可行的商业模式,在保证广告主的广告曝光量的前提下最大化广告投放效果是一个有意义的研究方向。为此本文提出了一种保量推荐下的最优广告投放算法,既在性能和投放量方面满足要求,而且相比于之前部署的算法获得了广告投放效果的提升。具体研究成果和结论总结如下:
\begin{itemize}
	
	\item 我们提出了一个基于用户定位的广告投放服务系统,主要由定位和广告投放两大部分组成,之后我们详细介绍了整体系统架构,并着重说明了广告投放部分的系统架构。
	
	\item 针对用户作为信号发射源的定位问题,提出了基于接收信号强度的几何模型发射源定位算法,即提出一种新的优化目标函数,本质上是找到一个点使得它到所有圆的边缘的距离之和最小。同时,为了从理论上分析该方法的克拉美罗下界,我们提出了一种基于蒙特卡洛法的估计方法。为了确保我们的算法相比于经典的最大似然估计方法定位精度的提升具有统计意义,我们还推导了均方根误差的置信区间的计算方法。仿真结果表明在噪声较大的情况下,我们的算法能够取得更小的定位误差而且具有统计显著性,因为相比于无偏的最大似然估计,我们的算法在偏置和方差之间取得折中,即能更好的应对“过拟合”问题。最后我们也利用路测仪实地采集数据,实验结果表明我们的算法依然能取得更高的定位精度,说明我们的算法更适用于在真实环境中对发射源定位。
	
	\item 针对用户作为接收端且仅使用接收信号强度的定位问题,我们提出了基于曼哈顿切线距离的指纹定位算法。首先介绍了指纹定位中常用的K近邻算法以及常用的距离度量方式,并进一步分析了这些度量方式存在的问题,针对这些弊端提出了指数变换法和将切线距离应用于指纹定位中。在介绍切线距离及其背后的思想后,我们阐释了如何将切线距离应用于指纹定位中,并突破原来只有基于欧氏距离的切线距离的束缚,提出了基于曼哈顿距离的切线距离,以及在计算复杂度和定位精度之间折中的近似曼哈顿切线距离。我们在清华大学校园内三处不同信道特征的地方利用路测仪实地采集数据,并在数据集上开展实验。实验结果表明曼哈顿切线距离在所有三个数据集上均取得最小的定位误差,而近似曼哈顿切线距离则以小得多的计算复杂度取得了仅次于曼哈顿切线距离的定位精度。
	
	\item 针对保量推荐下的广告投放问题,我们提出了一种最优广告投放算法。在将实际问题抽象为数学优化问题后,利用凸优化方法中的KKT条件以及对偶问题推导出可以在线上高效投放广告的算法,同时还详细阐述了一些工程实现上的细节,以增加算法的实用性。在部署到线上之前,仿真实验证实了我们的算法的缺投量近似服从零均值正态分布,投放效果明显优于简单贪心算法。之后我们在中国最大的短视频平台之一——快手上开展了为期一个月的线上A/B测试,结果显示在计算时间和投放量等性能指标上我们的算法达标;在投放效果指标方面,我们的算法在最主要的指标上优于原先部署的算法而在其他次要指标上持平。最终得出结论:新算法在保证投放量的情况下可以获得比原先算法更好的投放效果。
\end{itemize}

\section{进一步工作展望}

虽然本论文的研究工作力求能够全面、深入的分析所涉及的问题,但是由于作者时间和能力的限制,仍有一些更深入、更本质的问题需要研究分析。
\begin{itemize}
	\item 在基于接收信号强度的几何模型发射源定位算法中,我们所提出的算法是一个非凸优化问题,这导致需要多次求解选取损失函数最小的一次,不仅时间开销大,而且还无法保证找到最优解。作为基线的最大似然估计也有同样的问题,不过已经有大量研究致力于将原问题转化为一个凸优化问题并力求减少由此带来的算法的解与真实最优解之间的差距。后续可以参考最大似然法的相关做法,应用于几何模型算法中,在定位误差少量增加的情况下减少计算时间。除此之外,在我们的算法中,路径损耗系数$\alpha$是需要提前设定好的参数,不能自适应的学习,从而会使得定位精度下降。如何自适应的学习$\alpha$也是一个可行的研究点。
	
	\item 在基于曼哈顿切线距离的指纹定位算法中,我们通过实测数据发现,在三个数据集上曼哈顿距离的定位精度均高于欧氏距离。首先需要确定这种现象是否具有普适性,如果答案是肯定的,则分析导致这种现象的原因将会具有重要意义。
	
	\item 在保量推荐下的最优广告投放算法中,我们的优化目标是投放广告的总分数最高,其中分数是关注率和点击率的线性组合。但是总最后的实验结果看来只有关注率提升了,点击率却几乎没有变化,而且一个月内几乎每天都是如此。为何唯独出现这种现象,而没有出现关注率不变、点击率提升或者二者皆提升的现象,是一个值得深入研究的点。
\end{itemize}



