\chapter{基于接收信号强度的几何模型发射源定位算法}
\label{cha:transmitter}

本章将要介绍基于接收信号强度的几何模型发射源定位算法,即用户作为信号发射源时实现对用户定位的算法。具体应用场景可以描述为:一个用户终端以恒定功率发射信号(比如蜂窝网或WLAN中用户上行通信),通信范围内的多个WAP(基站或是无线路由器)接收到信号并测得RSS,最后根据这些RSS实现对用户定位。

\section{路径损耗模型和估计器}

\subsection{路径损耗模型}

不妨设总共有$N$个WAP即信号接收者,则第$i$个接收者的RSS可用下式表示~\cite{erceg1999empirically}:
\begin{equation}
{P_i} = C - 10\alpha {\log _{10}}({\left\| {{\bm{\theta}} - {{\bm{\mathrm{x}}_i}}} \right\|_2}) + {n_i}, i = 1,2,...,N,\label{eq:rss}
\end{equation}
其中$P_i$是接收者的RSS,$\bm{\theta}$是发射源即用户的位置,$\alpha$是路径损耗指数,$C$是一个与$\bm{\theta}$ 和 $\bm{\mathrm{x}}_i$无关的常数,$n_i$则是均值为$0$、方差为${\sigma_i}^2$的高斯白噪声。为了简化表达,我们用下式代表发射源与接收端之间的距离:
\begin{equation}
d_i = {\left\| {{\bm{\theta}} - {{\bm{\mathrm{x}}_i}}} \right\|_2}, \forall i. \label{eq:d}
\end{equation}
因为$C$是常数,所以它可以通过与另一个接收者的RSS相减被消除,即:
\begin{equation}
{P_{ij}} = 10\alpha ({\log _{10}}d_j - {\log _{10}}d_i) + {n_{ij}}, i \neq j, \label{eq:diff_rss}
\end{equation}
其中$P_{ij} = P_i - P_j$,$n_{ij}$变为$0$均值、方差为$\sigma_i^2 + \sigma_j^2$的高斯白噪声。注意到,因为我们没有锚点,因此这里让$i$作为RSS最大的接收者的下标。为了使表达更清晰,在接下来的部分我们将用$\bm{\mathrm{x}}_M, d_M$代表$\bm{\mathrm{x}}_i, d_i$。

\subsection{传统最大似然估计器}

传统最大似然(Maximum Likelihood, ML)估计器通过求解如下优化问题从而找到定位$\bm{\hat{\theta}}$:
\begin{equation}
\widehat {\bm{\theta }} = \mathop {\arg \min }\limits_{\bm{\theta }} \sum\limits_{j = 1}^N {{{[{P_{Mj}} - 10\alpha ({{\log }_{10}}d_j - {{\log }_{10}}d_M)]}^2}}, j \neq M. \label{eq:ml}
\end{equation}
因为这是一个非凸优化问题,许多凸优化估计器(比如半正定规划)和线性最小二乘估计被提出,正如第\ref{cha:intro}章所述。因为ML估计器是作为算法比较基准线的,我们需要选择一个能够取得最小误差的作为基准,因此直接优化非凸的非线性最小二乘问题。为了尽可能找到最优解,我们会多次随机初始点,选择损失函数最小的作为最终定位。

\subsection{基于几何的模型}

S. Wang 和 R. Inkol~\cite{wang2011near} 提出了一种线性最小二乘估计器,在推导过程中发现每两个接收端RSS可以形成一个发射源坐落在其上的圆。但是这个结论没有被作者直接用来优化,而是提出了一种线性最小二乘法。接下来我们将用这个知识得到基于几何的模型(Geometry-Based Model, GBM)。

如果我们忽略\eqref{eq:diff_rss}中的噪声项,可以得到:
\begin{equation}
{P_{ij}} - 5\alpha {\log _{10}}({\dfrac{d_j}{d_i}})^2 = 0, i \neq j. \label{eq:ignore}
\end{equation}
令:
\begin{equation}
\gamma = 10^{\frac{P_{ij}}{5\alpha}}, \label{eq:gamma}
\end{equation}
则\eqref{eq:ignore}可以被改写为:
\begin{equation}
{\left\| {\bm{\theta}  - \bm{\mathrm{x}}_j} \right\|_2}^2 = {\gamma _{ij}}{\left\| {\bm{\theta}  - \bm{\mathrm{x}}_i} \right\|_2}^2. \label{eq:10exp}
\end{equation}
如果 $\gamma _{ij} \neq 1$,即两个接收端的RSS不相等,\eqref{eq:10exp}可以被改写为:
\begin{equation}
{\left\| {\bm{\theta}  - \frac{\bm{\mathrm{x}}_j - \gamma_{ij}\bm{\mathrm{x}}_i}{1 - \gamma_{ij}}} \right\|_2}^2 = \frac{\gamma_{ij}{d_{ij}}^2}{(1 - \gamma_{ij})^2}, \label{eq:circle}
\end{equation}
其中$d_{ij} = \left\| \bm{\mathrm{x}}_i  - \bm{\mathrm{x}}_j \right\|_2$。显然这是一个$\bm{\theta}$坐落在其边缘上的圆,而圆心和半径分别为:
\begin{equation}
\bm{\mathrm{o}}_j = \frac{\bm{\mathrm{x}}_j - \gamma_{ij}\bm{\mathrm{x}}_i}{1 - \gamma_{ij}}, {r_j}^2 = \frac{\gamma_{ij}{d_{ij}}^2}{(1 - \gamma_{ij})^2}. \label{eq:oR}
\end{equation}

让$i$座位最大RSS接收端的下标(如果有多于1个最大RSS,则随机挑选一个而提出其余的)并用$M$替换$i$。假设\eqref{eq:circle}被下面的加性噪声影响:
\begin{equation}
{\left\| {\bm{\theta}  - o_j} \right\|_2}^2 = d_j + n_j, j \neq M, \label{eq:GBMnoise}
\end{equation}
其中$n_j$是一个服从均值为$0$、方差为$\sigma_j^2$的截断高斯分布的噪声。因为$\bm{\theta}$是实数,因此\eqref{eq:GBMnoise}等式右边必须大于等于$0$,即$n_j$的取值范围是 $[-d_j, +\infty)$。所以我们建立了一个带有噪声$n_j$的模型,服从截断高斯分布。注意到这里的$n_j$与\eqref{eq:rss}中的噪声无关,因为这是两个独立的模型。

利用上述模型,我们可以定义一个服从截断高斯分布的随机变量$X_j = \left\|{\bm\theta} - \bm{\mathrm{o}}_j\right\|_2$,其概率密度函数是:
\begin{equation}
f(x_j) = \frac{1}{C(r_j)}\exp(-\frac{(x_j - r_j)^2}{2{\sigma_j}^2})I[0, +\infty), \label{eq:pdf}
\end{equation}
其中 $C(r_j) = \int_0^{ + \infty }\exp(-\frac{(x_j - r_j)^2}{2{\sigma_j}^2})dx_j$是归一化系数,$I[0, +\infty)$是指示函数,在$[0, +\infty)$范围内取值$1$其余为$0$。

因此,最终的损失函数和优化问题可以表示为:
\begin{equation}
\widehat {\bm{\theta }} = \mathop {\arg \min }\limits_{\bm{\theta }} \frac{1}{2}\sum\limits_{j = 1}^{N^-} {\frac{1}{\sigma_j^2}(\left\|{\bm\theta} - \bm{\mathrm{o}}_j\right\|_2 - r_j)^2}, \label{eq:loss}
\end{equation}
其中$N^-$代表剩余的接收端数量。因为$C(r_j)$和$\bm\theta$无关,因此在优化时可以忽略。

当假设所有接收端的噪声方差$\sigma_j^2$相同时,该优化问题就变成找到一个点,使得它到所有圆边缘的距离之和最小。因此这个优化问题有着清楚的几何意义,于是我们把\eqref{eq:loss}称为基于几何的模型(Geometry-Based Model, GBM)。

因为无法证明GBM的凸性,我们将采用和ML中相同的方法求解,即多次随机挑选初始点后选取使得损失函数最小的结果。

后续的仿真结果表明GBM估计器是有偏的。但是注意到,我们的目标是最小化均方根误差(Mean Square Error, MSE)而MSE可以写作:
\begin{equation}    \label{eq:mse}
\begin{split}
MSE &= \mathrm{E}[(\widehat{\theta} - \theta)^2]\\
&= (\mathrm{E}[\widehat{\theta}] - \theta)^2 + \mathrm{E}[(\widehat{\theta} - \mathrm{E}[\widehat{\theta}])^2]\\
&= bias^2 + variance.
\end{split}
\end{equation}
即MSE是由估计器的偏差的平方和方差两部分组成,这也是机器学习领域经常出现的经典问题。若一个估计器有很小的偏差却有很大的方差则被称为“过拟合”(Over-Fitting)。在仿真和实测数据验证部分,我们会展示传统最大似然估计器相比于GBM是过拟合的。

\section{克拉美罗下界(Cram\'{e}r-Rao Lower Bound, CRLB)}

CRLB是估计器的理论误差下界,虽然不一定能够达到,但是对其进行分析有助于研究估计器的性能。

\subsection{无偏估计的CRLB}

对于路径损耗模型,因为第$i$个接收端的RSS的分布服从$p_i \sim \mathcal{N}(-10\alpha\mathrm{log}_{10}{d_i} -C, {\sigma_i}^2)$,因此费舍信息(Fisher Information)可表示为:
\begin{equation}
I(\bm{\theta}) = (\frac{10\alpha}{\sigma\mathrm{ln}10})^2\sum\limits_{i = 1}^N(\frac{{\bm\theta} - \bm{\mathrm{x}}_i}{{\left\|{\bm\theta} - \bm{\mathrm{x}}_i\right\|_2}^2})(\frac{{\bm\theta} - \bm{\mathrm{x}}_i}{{\left\|{\bm\theta} - \bm{\mathrm{x}}_i\right\|_2}^2})^T, \label{eq:fisher}
\end{equation}
其中$\sigma$是噪声的标准差。如果估计器是无偏的,则协方差矩阵的CRLB为:
\begin{equation}
\begin{split}
CRLB(\bm{\theta}) &= I(\bm{\theta})^{-1} \label{eq:unbias_cov}\\
&\le Cov(\widehat{\bm{\theta}}_{ML}).
\end{split}
\end{equation}

因为最大似然估计器是渐进无偏的~ \cite{kay1993fundamentals},根据\eqref{eq:mse},最大似然法的RSME的估计下界可以表示为:
\begin{equation}
RMSE \ge \sqrt{Tr(CRLB)},\label{eq:rmse_ml}
\end{equation}
其中$Tr(\cdot)$是矩阵的迹,即对角线元素累加。

\subsection{有偏估计的CRLB}

假设GBM估计器的期望是$\bm\phi(\bm{\theta})$,如果估计器有偏,则协方差矩阵的CRLB可由下式所得:
\begin{equation}
\begin{split}
CRLB_{GBM}(\bm{\theta}) &= {\frac{\partial\bm\phi(\bm{\theta})}{\partial \bm\theta}}I(\bm{\theta})^{-1}{\frac{\partial\bm\phi(\bm{\theta})}{\partial \bm\theta}}^T\\
& \le Cov(\widehat{\bm{\theta}}_{GBM}),
\end{split}
\end{equation}
其中$\frac{\partial\bm\phi(\bm{\theta})}{\partial \bm\theta}$是雅可比(Jacobian)矩阵。注意到如果估计器是无偏的,则$\bm\phi(\bm{\theta}) = \bm{\theta}$即$\frac{\partial\bm\phi(\bm{\theta})}{\partial \bm\theta} = \bm{I}$,而这与\eqref{eq:unbias_cov}一致。

根据\eqref{eq:mse},GBM的RMSE的下界可表示为:
\begin{equation}
RMSE\! \ge\! \sqrt{{(\bm\phi(\bm{\theta})\! -\! \bm{\theta})}^T(\bm\phi(\bm{\theta})\! -\! \bm{\theta})\! +\! \mathrm{Tr}(CRLB_{GBM}(\bm{\theta}))}.\label{eq:rmse_gbm}
\end{equation}

直接计算$\bm\phi(\bm{\theta})$ 和 ${\frac{\partial\bm\phi(\bm{\theta})}{\partial \bm\theta}}$较困难,但是我们知道接收者RSS的分布,因此蒙特卡洛法适用于该问题。

首先,$\bm\phi(\bm{\theta})$比较容易用下式得到:
\begin{equation}
\begin{split}
\bm\phi(\bm{\theta}) &= \mathrm{E}[\widehat{\bm{\theta}}]\\
&\approx \frac{\sum\limits_{k = 1}^K{\widehat{\bm{\theta}}}}{K},
\end{split}
\end{equation}
其中,$K$是仿真次数。${\frac{\partial\bm\phi(\bm{\theta})}{\partial \bm\theta}}$可以直接用有限差分法得到,但是这可能需要大量时间计算,因为我们需要在二维平面内多次用不同的$\bm\theta$进行仿真。

接下来,我们将推导一种不需要用不同的$\bm\theta$仿真就可以得到${\frac{\partial\bm\phi(\bm{\theta})}{\partial \bm\theta}}$的方法。首先 $\bm\phi(\bm{\theta})$可以用如下公式表示:
\begin{equation}
\begin{split}
&Let \quad f(\bm{\theta}) = {\prod\limits_{i=1}^N}(2\pi{\sigma_i}^2)^{-\frac{1}{2}}e^{-\frac{(p_i + 10\alpha\mathrm{log}_{10}{d_i} + C)^2}{2{\sigma_i}^2}},\\
&then \quad \bm\phi(\bm{\theta}) = \iiint_{\mathbb{R}^N}\widehat{\bm{\theta}}f(\bm{\theta})dp_1...dp_N,
\end{split}
\end{equation}
其中$d_i$由\eqref{eq:d}定义。若$\widehat{\bm{\theta}}f(\bm{\theta})$ 和 $\frac{\partial \widehat{\bm{\theta}}f(\bm\theta)}{\partial \bm{\theta}}$是连续的,$\iiint_{\mathbb{R}^N}\widehat{\bm{\theta}}f(\bm\theta)$ 和 $\iiint_{\mathbb{R}^N}\frac{\partial \widehat{\bm{\theta}}f(\bm\theta)}{\partial \bm{\theta}}$一致收敛,则
\begin{equation}
\begin{split}
{\frac{\partial\bm\phi(\bm{\theta})}{\partial \bm\theta}} &= \frac{\partial\mathrm{E}[\widehat{\bm{\theta}}]}{\partial \bm{\theta}}\\
&= \frac{\partial}{\partial \bm{\theta}}\iiint_{\mathbb{R}^N} \widehat{\bm{\theta}}f(\bm{\theta}) dp_1...dp_N\\
&= \iiint_{\mathbb{R}^N} \frac{\partial \widehat{\bm{\theta}}f(\bm{\theta})}{\partial \bm{\theta}} dp_1...dp_N\\
&= \iiint_{\mathbb{R}^N} \sum\limits_{i=1}^N \widehat{\bm{\theta}}(\frac{-10\alpha}{{\sigma_i}^2\mathrm{ln}10}\frac{{\bm\theta} - \bm{\mathrm{x}}_i}{{\left\|{\bm\theta} - \bm{\mathrm{x}}_i\right\|_2}^2})^Tf(\bm{\theta}) dp_1...dp_N\\
&= \mathrm{E}[\sum\limits_{i=1}^N\widehat{\bm{\theta}}(\frac{-10\alpha}{{\sigma_i}^2\mathrm{ln}10}\frac{{\bm\theta} - \bm{\mathrm{x}}_i}{{\left\|{\bm\theta} - \bm{\mathrm{x}}_i\right\|_2}^2})^T]. \label{eq:mc}
\end{split}
\end{equation}
需要注意$\widehat{\bm{\theta}}$与$\bm{\theta}$无关,而是关于$(\bm{\mathrm{x}}_1,...,\bm{\mathrm{x}}_N)^T$ 和 $(P_1,...,P_N)^T$的函数。

根据\eqref{eq:mc}的结果,我们可以通过计算$\sum\limits_{i=1}^N\widehat{\bm{\theta}}(\frac{-10\alpha}{{\sigma_i}^2\mathrm{ln}10}\frac{{\bm\theta} - \bm{\mathrm{x}}_i}{{\left\|{\bm\theta} - \bm{\mathrm{x}}_i\right\|_2}^2})^T$的经验平均值来利用蒙特卡洛法近似估算 ${\frac{\partial\bm\phi(\bm{\theta})}{\partial \bm\theta}}$,而不需要有限差分,这会大量节省仿真时间。

\section{RMSE的置信区间}






