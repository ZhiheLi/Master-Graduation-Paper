\begin{resume}

  \resumeitem{个人简历}

  1993 年 11 月 20 日出生于辽宁省沈阳市。

  2012 年 9 月考入清华大学电子工程系电子信息科学类专业,2016 年 7 月本科毕业并获得工学学士学位。

  2016 年 9 月免试进入清华大学电子工程系攻读信息与通信工程学位至今。

  \researchitem{发表的学术论文} % 发表的和录用的合在一起

  % 1. 已经刊载的学术论文(本人是第一作者,或者导师为第一作者本人是第二作者)
  \begin{publications}
    \item Li, Zhihe, Xiaofeng Zhong, and Zechen Cui. “Evaluating forecasting algorithm of realistic datasets based on machine learning.” Proceedings of the 2nd International Conference on Innovation in Artificial Intelligence. ACM, 2018. (EI 收录, 检索号:20183905853244.)
    \item Li, Zhihe, Xiaofeng Zhong, and Jie Wei. “A Novel Geometry-Based Model for Localization Based on Received Signal Strength.” 2018 IEEE 87th Vehicular Technology Conference (VTC Spring). IEEE, 2018. (EI 收录, 检索号:20183205660551.)
    \item Li, Zhihe, et al. “The Application of Manhattan Tangent Distance in Outdoor Fingerprint Localization.” 2018 IEEE Global Communications Conference (GLOBECOM). IEEE, 2018. 
  \end{publications}

  % 2. 尚未刊载,但已经接到正式录用函的学术论文(本人为第一作者,或者
  %    导师为第一作者本人是第二作者)。
  % \begin{publications}[before=\publicationskip,after=\publicationskip]
    
  % \end{publications}

  % 3. 其他学术论文。可列出除上述两种情况以外的其他学术论文,但必须是
  %    已经刊载或者收到正式录用函的论文。
  \begin{publications}
    \item Chen, Huangqing, et al. "Energy-Saving Algorithm with Dimension Reduction on the Uplink for Multimedia Push." 2017 IEEE 86th Vehicular Technology Conference (VTC-Fall). IEEE, 2017. (EI 收录, 检索号:20181605013949 .)
  \end{publications}

  % \researchitem{研究成果} % 有就写,没有就删除
  % \begin{achievements}
    
  % \end{achievements}

\end{resume}
